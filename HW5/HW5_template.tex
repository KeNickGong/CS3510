\documentclass{article}
%\usepackage[T1]{fontenc}
%\usepackage{amssymb, amsmath, graphicx, subfigure, enumerate}
%\usepackage{amsthm,alltt} 
\usepackage[margin=1.25in]{geometry} %geometry (sets margin) and other useful packages
\usepackage{graphicx,ctable,booktabs}
\usepackage{mathtools}
\usepackage[boxed]{algorithm2e}
\usepackage{mathdots}
\usepackage{fancyhdr} %Fancy-header package to modify header/page numbering
\usepackage{cleveref}

\setlength{\oddsidemargin}{.25in}
\setlength{\evensidemargin}{.25in}
\setlength{\textwidth}{6in}
\setlength{\topmargin}{-0.4in}
\setlength{\textheight}{8.5in}



\newcommand{\heading}[6]{
  \renewcommand{\thepage}{\arabic{page}} % used to be #1-\arabic{page}
  \noindent
  \begin{center}
  \framebox{
    \vbox{
      \hbox to 5.78in { \textbf{#2} \hfill #3 }
      \vspace{4mm}
      \hbox to 5.78in { {\Large \hfill #6  \hfill} }
      \vspace{2mm}
      \hbox to 5.78in { \textit{{Name: }} }
    }
  }
  \end{center}
  \vspace*{4mm}
}

%Redefining sections as problems
\makeatletter
\newenvironment{problem}{\@startsection
       {section}
       {2}
       {-.2em}
       {-3.5ex plus -1ex minus -.2ex}
       {2.3ex plus .2ex}
       {\pagebreak[3]%forces pagebreak when space is small; use \eject for better results
       \large\bf\noindent{Problem }
       }
       }
       %{%\vspace{1ex}\begin{center} \rule{0.3\linewidth}{.3pt}\end{center}}
       %\begin{center}\large\bf \ldots\ldots\ldots\end{center}}
\makeatother


\newtheorem{theorem}{Theorem}[section]
\newtheorem{definition}[theorem]{Definition}
\newtheorem{remark}[theorem]{Remark}
\newtheorem{lemma}[theorem]{Lemma}
\newtheorem{corollary}[theorem]{Corollary}
\newtheorem{proposition}[theorem]{Proposition}
\newtheorem{claim}[theorem]{Claim}
\newtheorem{observation}[theorem]{Observation}
\newtheorem{fact}[theorem]{Fact}
\newtheorem{assumption}[theorem]{Assumption}

%\newenvironment{proof}{\noindent{\bf Proof:} \hspace*{1mm}}{
% \hspace*{\fill} $\Box$ }
%\newenvironment{proof_of}[1]{\noindent {\bf Proof of #1:}
% \hspace*{1mm}}{\hspace*{\fill} $\Box$ }
%\newenvironment{proof_claim}{\begin{quotation} \noindent}{
% \hspace*{\fill} $\diamond$ \end{quotation}}

\newcommand{\problemset}[3]{\heading{#1}{\classname}{#2}{\studentname}{#3}{Problem Set #1}} % Don't change this line
%%%%%%%%%%%%%%%%%%%%%%%%%% Change this stuff below, don't change the line above this one
\newcommand{\problemsetnum}{1}            % problem set number
\newcommand{\duedate}{Due: Jan. 15, 2018, 8am EST} % problem set deadline
\newcommand{\studentname}{Student Name: }      % name of student (i.e., you)
\newcommand{\classname}{Name:   }
%\newcommand{\instructor}{Prof. Eric Vigoda}
%%%%%%%%%%%%%%%%%%%%%%%%%%

\pagestyle{fancy}
%\addtolength{\headwidth}{\marginparsep} %these change header-rule width
%\addtolength{\headwidth}{\marginparwidth}
\lhead{\classname} %Problem \thesection}
\chead{} 
\rhead{\thepage} 
%\lfoot{\small\scshape \classname}
%\cfoot{} 
%\rfoot{\footnotesize PS \#\problemsetnum} 
\renewcommand{\headrulewidth}{.3pt} 
\renewcommand{\footrulewidth}{.3pt}
\setlength\voffset{-0.25in}
\setlength\textheight{648pt}


\newcommand{\sit}{\shortintertext}
\newcommand\deq{\mathrel{\overset{\makebox[0pt]{\mbox{\normalfont\tiny\sffamily def}}}{=}}}
\newcommand{\ones}{\mathbbm{1}}
\newcommand{\e}{\vec{e}}
\newcommand{\tr}{\text{tr}}
\newcommand{\bs}{\boldsymbol}
\mathchardef\mhyphen="2D
\newcommand{\C}{\mathbb{C}}
\newcommand{\R}{\mathbb{R}}
\newcommand{\II}{\mathcal{I}}
\newcommand{\FF}{\mathcal{F}}
\newcommand{\X}{\mathcal{X}}
\newcommand{\Y}{\mathcal{Y}}
\newcommand{\ra}{\rightarrow}
\newcommand{\Ra}{\Rightarrow}
\newcommand{\PP}{\mathbb{P}}
\newcommand{\sse}{\subseteq}
\newcommand{\eps}{\epsilon}
\newcommand{\N}{\mathcal{N}}
\newcommand{\poly}{\textup{poly}}

\newcommand{\dom}{\textup{dom}}

\renewcommand{\thesubsection}{\thesection.\roman{subsection}}


% auto sized delimiters
\newcommand{\Br}[1]{\left\{#1\right\}}
\newcommand{\br}[1]{\left[#1\right]}
\newcommand{\pr}[1]{\left(#1\right)}
\newcommand{\ceil}[1]{\left\lceil#1\right\rceil}
\newcommand{\floor}[1]{\left\lfloor#1\right\rfloor}
\newcommand{\abs}[1]{\left|#1\right|}
\newcommand{\sgn}{\textup{sgn}}

%default delimiter for Pr and E
\DeclarePairedDelimiter{\defaultDelim}{[}{]}

\DeclareMathOperator{\capPr}{Pr}
\renewcommand{\Pr}[2][]{\capPr_{#1}\defaultDelim*{#2}}
\DeclareMathOperator{\capE}{E}
\newcommand{\E}[2][]{\capE_{#1}\defaultDelim*{#2}}
\DeclareMathOperator{\capVar}{Var}
\newcommand{\Var}[2][]{\capVar_{#1}\defaultDelim*{#2}}

\newcommand{\vs}{\vspace{.1in}}
\newcommand{\vB}{\vspace{.3in}}

%\DeclareMathOperator*{}{} puts subscript below


%%%%%%%%%%%%%%%%%%%%%%%%%%%%%%%%%%%%%%%%%%%%%%%%%
\begin{document}
{\bf \noindent Homework 5. \\ Due: Thursday, July 15, 2021 before 11:59PM via Gradescope. Late submission with no penalty by Friday, July 16, 2021 before 11:59AM.}

\begin{problem}{(SAT variant).}
 
%Type your answer here. 
Solution:\\
Given an input of 3-CNF that each literals appear no more than 3 times
we check that the problem is in NP checking if any one of the 3 literals in
n claues equal to 1. This takes runtime \(O(3n)\) and this is polynomial time.
So the problem is in class np.\\
\\
We now reduce 3-SAT to our problem. We first map a 3-SAT instance to the instance of our problem.\\
For each variable in the 3-SAT instance that appears in more than 3 claues, let the variable be x, we replace its kth appearance by \(x_k\).\\
For example, variable x with 4 appearances in 4 clauses will be replaced by \(x_1,x_2,x_3,x_4\).\\
Then we add clauses:\\
\[(\bar{x}_1 \vee x_2)\wedge(\bar{x}_2 \vee x_3)...(\bar{x}_k \vee x_1)\]
This is to make sure that each sub variable we create has the same value. As the added claues show, if x is originally true, by making \(x_1\) true, \(x_2\) has to be true according to the second clause, and the rest of the variables are forced to be true. vice versa.\\
The adding clause part has runtime O(m) because there are at most m clauses and thus at most m variables.\\
\\
We repeat this for each variable that appear in more than 3 clauses.\\
This mapping has runtime O(nm) which is in polynomial time, n is the number of variables.\\
\\
The 3-SAT problem has a satisfying solution if and only if the reduced problem has a solution:\\
Given a truth assignment of the 3-SAT problem, simply setting the created variables \(x_i\) to x in our problem gives us a truth solution.\\
Given a truth assignment of our problem, we can just put those variables back together, this takes O(m) which is polynomial time.\\
\\
This concludes the reduction.\\

\end{problem}


\newpage

\begin{problem}{(Max diameter spanning tree).}

%Type your answer here. 	
Solution:\\
We reduce from Rudrata path to the MDST problem. We map the instance graph G and pass it to MDST. This has runtime O(n+m) and is polynomial.\\
We also pass in g = |V|-1 which V is the number of vertices. The reason to pass in |V|-1 is that rudrata path requires the path to go through all vertices while visiting each vertex only once, and that forces the path to go through |V|-1 edges which make the length of the path.\\
\\
We show that there is a solution for Rudrata path if and only if there is one for the MDST.\\
Note that the length of the Rudrata path has to be |V|-1, so if a rudrata path exist, the rudrata path will be the diameter making the diameter's length |V|-1.\\
Similarily, if MDST(G, |V|-1) has a solution, the diameter has length |V|-1 and is a path.\\
\\
This concludes the reduction.\\

\end{problem}



\end{document}



